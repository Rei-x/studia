\documentclass[12pt,a4paper]{article}
\usepackage[utf8]{inputenc}
\usepackage[T1]{fontenc}
\usepackage[polish]{babel}
\usepackage{amsmath}
\usepackage{amssymb}
\usepackage{graphicx}
\usepackage[a4paper, margin=2.5cm]{geometry}
\usepackage{hyperref}
\usepackage{listings}
\usepackage{float}
\usepackage{caption}
\usepackage{booktabs} % For professional looking tables
\usepackage{xcolor}   % For colored text, if needed
\usepackage{multicol} % For multi-column layout if needed for tables

% Define colors (from user's sample)
\definecolor{codegreen}{rgb}{0,0.6,0}
\definecolor{codegray}{rgb}{0.5,0.5,0.5}
\definecolor{codepurple}{rgb}{0.58,0,0.82}
\definecolor{backcolour}{rgb}{0.95,0.95,0.92}

% Define PDDL listing style
\lstdefinestyle{pddlstyle}{
    backgroundcolor=\color{backcolour},   
    commentstyle=\color{codegreen},
    keywordstyle=\color{blue}\bfseries, % Main PDDL keywords
    numberstyle=\tiny\color{codegray},
    stringstyle=\color{codepurple}, % For object names or parameters if styled as strings
    basicstyle=\ttfamily\footnotesize,
    breakatwhitespace=false,         
    breaklines=true,                 
    captionpos=b,                    
    keepspaces=true,                 
    numbers=none, % Line numbers can be enabled if desired, e.g., left               
    numbersep=5pt,                  
    showspaces=false,                
    showstringspaces=false,
    showtabs=false,                  
    tabsize=2,
    morekeywords=[1]{define, domain, problem, :requirements, :strips, :typing, 
                  :negative-preconditions, :equality, :constants, :predicates, 
                  :action, :parameters, :precondition, :effect, 
                  :objects, :init, :goal, and, or, not}, % Primary keywords
    morekeywords=[2]{package, location, vehicle, transport-mode, university, drone, student,
                  room, robot, ball, arm}, % Common type names (styled differently)
    keywordstyle=[2]\color{codepurple}, % Style for type names
    sensitive=false, % PDDL keywords are typically case-insensitive
    comment=[l]{;} % PDDL comments start with ;
}
\lstset{style=pddlstyle} % Apply this style globally for listings

\newcommand{\code}[1]{\texttt{\detokenize{#1}}} % Helper command for inline code

\title{\LARGE \textbf{Sztuczna inteligencja i inżynieria wiedzy}\\
\large Lista nr 3: Planowanie z wykorzystaniem PDDL}
\author{Bartosz Gotowski}
\date{\today}

\begin{document}

\maketitle
\begin{center}
    \textit{Sprawozdanie z realizacji zadań dotyczących planowania automatycznego \\ z wykorzystaniem języka PDDL.}
\end{center}
\vspace{1cm}

\tableofcontents
\clearpage

\section{Wprowadzenie}
Celem niniejszej listy zadań było praktyczne zapoznanie się z językiem PDDL (Planning Domain Definition Language) oraz jego zastosowaniem w rozwiązywaniu problemów planowania automatycznego. PDDL jest standardowym językiem służącym do opisu problemów planowania w dziedzinie sztucznej inteligencji, umożliwiającym definiowanie dziedzin (dostępnych akcji, predykatów, typów) oraz konkretnych problemów (obiektów, stanu początkowego, celu).

W ramach listy zrealizowano trzy zadania o różnym stopniu złożoności, modelując je w PDDL i wykorzystując planer Fast Downward (w ramach systemu DELFI) do znalezienia rozwiązania:
\begin{enumerate}
    \item Problem logistyczny transportu kart graficznych między politechnikami przy użyciu dronów i studentów.
    \item Problem robota sprzątającego, który ma za zadanie odkurzyć wszystkie pomieszczenia.
    \item Problem robota z dwoma ramionami, przenoszącego piłki między pokojami.
\end{enumerate}
Niniejsze sprawozdanie przedstawia teoretyczne podstawy PDDL, szczegółowy opis sformułowania każdego z problemów, ich implementację w PDDL, a także analizę wyników uzyskanych z planera, w tym wygenerowane plany działania oraz metryki wydajnościowe.

\section{Podstawy Teoretyczne Języka PDDL}
Język PDDL (Planning Domain Definition Language) to formalny język opisu problemów planowania, szeroko stosowany w badaniach nad sztuczną inteligencją, zwłaszcza w dziedzinie planowania symbolicznego. PDDL korzysta z notacji prefiksowej, inspirowanej językiem LISP (tzw. S-wyrażenia).

Definicja problemu w PDDL składa się z dwóch głównych części, często zapisywanych w oddzielnych plikach:
\begin{enumerate}
    \item \textbf{Definicja dziedziny (\code{domain.pddl}):} Opisuje ogólne zasady świata, w którym działa agent. Zawiera:
    \begin{itemize}
        \item \code{:requirements}: Określa, jakie rozszerzenia PDDL są używane (np. \code{:strips} dla podstawowego modelu, \code{:typing} dla typowania obiektów, \code{:negative-preconditions} dla warunków negatywnych).
        \item \code{:types}: Definiuje typy obiektów i ich hierarchię (np. \code{samochod - pojazd}).
        \item \code{:constants}: Definiuje obiekty, które są stałe we wszystkich problemach danej dziedziny.
        \item \code{:predicates}: Deklaruje predykaty, czyli relacje lub właściwości, które mogą być prawdziwe lub fałszywe w danym stanie świata (np. \code{(na ?x ?y)} – obiekt ?x jest na obiekcie ?y).
        \item \code{:action}: Definiuje możliwe akcje, które agent może wykonać. Każda akcja zawiera:
        \begin{itemize}
            \item \code{:parameters}: Zmienne reprezentujące obiekty biorące udział w akcji, wraz z ich typami.
            \item \code{:precondition}: Warunki, które muszą być spełnione, aby akcja mogła zostać wykonana.
            \item \code{:effect}: Zmiany w stanie świata wynikające z wykonania akcji (dodanie lub usunięcie predykatów).
        \end{itemize}
    \end{itemize}
    \item \textbf{Definicja problemu (\code{problem.pddl}):} Opisuje konkretną instancję problemu do rozwiązania w ramach zdefiniowanej dziedziny. Zawiera:
    \begin{itemize}
        \item \code{(:domain <nazwa-dziedziny>)}: Wskazuje, której dziedziny dotyczy problem.
        \item \code{:objects}: Deklaruje konkretne obiekty istniejące w problemie, wraz z ich typami.
        \item \code{:init}: Opisuje stan początkowy świata poprzez listę predykatów, które są w nim prawdziwe.
        \item \code{:goal}: Opisuje stan docelowy, który agent ma osiągnąć, również poprzez listę predykatów.
    \end{itemize}
\end{enumerate}
Planery, takie jak Fast Downward, analizują te definicje, aby automatycznie wygenerować sekwencję akcji (plan), która transformuje stan początkowy w stan docelowy. Proces ten zwykle obejmuje parsowanie plików PDDL, normalizację zadania, a następnie przeszukiwanie przestrzeni stanów w poszukiwaniu rozwiązania.

\section{Sformułowanie Problemów}

\subsection{Zadanie 1: Logistyka Transportu Kart Graficznych}
\begin{itemize}
    \item \textbf{Opis:} Zadanie polega na przetransportowaniu dwóch kart graficznych (GPU1, GPU2) z Politechniki Wrocławskiej (PWr) do dwóch innych uczelni: Politechniki Warszawskiej (PW) i Akademii Górniczo-Hutniczej (AGH). Do dyspozycji są dwa środki transportu: dron z PWr oraz student z AGH. Każdy środek transportu ma swoje ograniczenia co do rodzaju trasy (powietrzna dla drona, drogowa dla studenta). Tematyka została wybrana z inspiracji własnymi doświadczeniami jako studenta Politechniki Wrocławskiej, łącząc elementy nowoczesnej technologii (drony) z codziennością akademicką.
    \item \textbf{Typy obiektów:} \code{package} (karta graficzna), \code{location} (z podtypem \code{university}), \code{vehicle} (z podtypami \code{drone}, \code{student}), \code{transport-mode} (stałe: \code{road}, \code{air}).
    \item \textbf{Obiekty:}
    \begin{itemize}
        \item Politechniki: \code{pwr}, \code{agh}, \code{pw}.
        \item Paczki: \code{gpu1}, \code{gpu2}.
        \item Pojazdy: \code{dron_pwr}, \code{student_agh}.
    \end{itemize}
    \item \textbf{Stan początkowy:}
    \begin{itemize}
        \item \code{gpu1} i \code{gpu2} znajdują się na \code{pwr}.
        \item \code{dron_pwr} znajduje się na \code{pwr}.
        \item \code{student_agh} znajduje się na \code{agh}.
        \item Zdefiniowane są połączenia drogowe i powietrzne (dwukierunkowe) między wszystkimi parami uczelni.
        \item \code{dron_pwr} może korzystać z tras powietrznych (\code{air}).
        \item \code{student_agh} może korzystać z tras drogowych (\code{road}).
    \end{itemize}
    \item \textbf{Cel:}
    \begin{itemize}
        \item \code{gpu1} znajduje się na \code{pw}.
        \item \code{gpu2} znajduje się na \code{agh}.
    \end{itemize}
\end{itemize}

\subsection{Zadanie 2: Robot Sprzątający}
\begin{itemize}
    \item \textbf{Opis:} Prosty robot ma za zadanie odwiedzić i posprzątać trzy pokoje.
    \item \textbf{Typy obiektów:} \code{room}, \code{robot}.
    \item \textbf{Obiekty:}
    \begin{itemize}
        \item Pokoje: \code{pokoj1}, \code{pokoj2}, \code{pokoj3}.
        \item Robot: \code{robby}.
    \end{itemize}
    \item \textbf{Stan początkowy:}
    \begin{itemize}
        \item Robot \code{robby} znajduje się w \code{pokoj1}.
        \item Wszystkie pokoje (\code{pokoj1}, \code{pokoj2}, \code{pokoj3}) są brudne (\code{dirty}).
    \end{itemize}
    \item \textbf{Cel:} Wszystkie pokoje są czyste (\code{clean}).
\end{itemize}

\subsection{Zadanie 3: Robot Przenoszący Piłki}
\begin{itemize}
    \item \textbf{Opis:} Robot wyposażony w dwa ramiona ma za zadanie przenieść cztery piłki z jednego pokoju do drugiego. Robot może poruszać się między pokojami oraz podnosić i odkładać piłki za pomocą ramion.
    \item \textbf{Typy obiektów:} \code{robot}, \code{room}, \code{ball}, \code{arm}.
    \item \textbf{Obiekty:}
    \begin{itemize}
        \item Pokoje: \code{room1}, \code{room2}.
        \item Robot: \code{robot} (jeden, nazwa generyczna).
        \item Piłki: \code{ball1}, \code{ball2}, \code{ball3}, \code{ball4}.
        \item Ramiona: \code{arm1}, \code{arm2}.
    \end{itemize}
    \item \textbf{Stan początkowy:}
    \begin{itemize}
        \item Robot \code{robot} oraz wszystkie cztery piłki znajdują się w \code{room1}.
        \item Oba ramiona robota (\code{arm1}, \code{arm2}) są puste (\code{arm-empty}).
    \end{itemize}
    \item \textbf{Cel:} Wszystkie cztery piłki znajdują się w \code{room2}.
\end{itemize}

\clearpage
\section{Opis Implementacji i Rozwiązania (Pliki PDDL)}

\subsection{Zadanie 1: Logistyka Transportu}
\subsubsection{Domena (\code{domain.pddl})}
\begin{lstlisting}[caption={Domena PDDL dla zadania 1 - Logistyka Transportu}, label={lst:zad1_domain}]
(define (domain transport-logistics)
  (:requirements 
    :strips 
    :typing 
    :negative-preconditions 
    :equality
  )

  (:types
    package
    location
    vehicle
    transport-mode - object

    university - location     ; University locations
    drone      - vehicle      ; Drone vehicles
    student    - vehicle      ; Student vehicles
  )

  (:constants
    road - transport-mode
    air  - transport-mode
  )

  (:predicates
    ; Object (package or vehicle) is at location
    (at ?obj - (either package vehicle) ?loc - location)
    
    ; Package is loaded in vehicle
    (in ?pkg - package ?veh - vehicle)
    
    ; Locations are connected by transport mode
    (connected ?loc1 - location ?loc2 - location ?mode - transport-mode)
    
    ; Vehicle can use transport mode
    (can-use ?veh - vehicle ?mode - transport-mode)
  )

  (:action load-package
    :parameters (?pkg - package 
                 ?veh - vehicle 
                 ?loc - location)
    :precondition (and
      (at ?pkg ?loc)
      (at ?veh ?loc)
    )
    :effect (and
      (in ?pkg ?veh)
      (not (at ?pkg ?loc))
    )
  )

  (:action unload-package
    :parameters (?pkg - package 
                 ?veh - vehicle 
                 ?loc - location)
    :precondition (and
      (in ?pkg ?veh)
      (at ?veh ?loc)
    )
    :effect (and
      (at ?pkg ?loc)
      (not (in ?pkg ?veh))
    )
  )

  (:action move-vehicle
    :parameters (?veh - vehicle 
                 ?from - location 
                 ?to - location 
                 ?mode - transport-mode)
    :precondition (and
      (at ?veh ?from)
      (connected ?from ?to ?mode)
      (can-use ?veh ?mode)
      (not (= ?from ?to)) ; prevent moving to the same location
    )
    :effect (and
      (at ?veh ?to)
      (not (at ?veh ?from))
    )
  )
)
\end{lstlisting}
\textbf{Opis Domeny:}
\begin{itemize}
    \item \textbf{Wymagania:} Wykorzystano podstawowe \code{:strips}, \code{:typing} do definiowania typów, \code{:negative-preconditions} (choć jawnie użyte tylko \code{not (= ...)} które jest częścią \code{:equality}) oraz \code{:equality} do porównywania obiektów.
    \item \textbf{Typy:} Zdefiniowano główne typy (\code{package}, \code{location}, \code{vehicle}, \code{transport-mode}) oraz ich podtypy (\code{university}, \code{drone}, \code{student}).
    \item \textbf{Stałe:} Tryby transportu \code{road} i \code{air}.
    \item \textbf{Predykaty:} \code{(at ?obj ?loc)} określa położenie obiektu, \code{(in ?pkg ?veh)} że paczka jest w pojeździe, \code{(connected ...)} połączenie między lokacjami, \code{(can-use ...)} zdolność pojazdu do użycia trybu transportu.
    \item \textbf{Akcje:}
    \begin{itemize}
        \item \code{load-package}: Załadowanie paczki do pojazdu w tej samej lokalizacji.
        \item \code{unload-package}: Wyładowanie paczki z pojazdu.
        \item \code{move-vehicle}: Przemieszczenie pojazdu między połączonymi lokacjami, jeśli pojazd może użyć danego trybu transportu. Warunek \code{(not (= ?from ?to))} zapobiega ruchowi w miejscu.
    \end{itemize}
\end{itemize}

\subsubsection{Problem (\code{problem.pddl})}
\begin{lstlisting}[caption={Problem PDDL dla zadania 1 - Logistyka Transportu}, label={lst:zad1_problem}]
(define (problem transport-task-politechniki)
  (:domain transport-logistics)

  (:objects
    pwr - university ; Politechnika Wroc{\l}awska
    agh - university ; AGH Krakow  
    pw  - university ; Politechnika Warszawska
    
    gpu1 - package   ; Graphics card 1
    gpu2 - package   ; Graphics card 2

    dron_pwr    - drone   ; Drone from Wroc{\l}aw
    student_agh - student ; Student from AGH
  )

  (:init
    (at gpu1 pwr) (at gpu2 pwr)
    (at dron_pwr pwr) (at student_agh agh)

    (connected pwr agh road) (connected agh pwr road)
    (connected agh pw road)  (connected pw agh road)
    (connected pwr pw road)  (connected pw pwr road)

    (connected pwr agh air) (connected agh pwr air)
    (connected agh pw air)  (connected pw agh air)
    (connected pwr pw air)  (connected pw pwr air)

    (can-use dron_pwr air)
    (can-use student_agh road)
  )

  (:goal
    (and
      (at gpu1 pw)  ; GPU 1 to Warsaw Tech
      (at gpu2 agh) ; GPU 2 to AGH Krakow
    )
  )
)
\end{lstlisting}
\textbf{Opis Problemu:}
Definiuje konkretne uniwersytety, paczki i pojazdy. Stan początkowy określa ich lokalizacje oraz połączenia (drogowe i powietrzne zdefiniowane jako dwukierunkowe) i zdolności pojazdów. Celem jest dostarczenie \code{gpu1} do \code{pw} i \code{gpu2} do \code{agh}.

\clearpage
\subsection{Zadanie 2: Robot Sprzątający}
\subsubsection{Domena (\code{domain.pddl})}
\begin{lstlisting}[caption={Domena PDDL dla zadania 2 - Robot Sprzątający}, label={lst:zad2_domain}]
(define (domain robot-world)
    (:requirements :strips :typing)
    (:types room robot)
    (:predicates
        (at ?r - robot ?p - room)
        (dirty ?p - room)
        (clean ?p - room)
    )
    (:action move
        :parameters (?r - robot ?from ?to - room)
        :precondition (at ?r ?from)
        :effect (and (not (at ?r ?from)) (at ?r ?to))
    )
    (:action clean
        :parameters (?r - robot ?p - room)
        :precondition (and (dirty ?p) (at ?r ?p))
        :effect (and (not (dirty ?p)) (clean ?p))
    )
)
\end{lstlisting}
\textbf{Opis Domeny:} Prosta domena z typami \code{room} i \code{robot}. Predykaty opisują pozycję robota oraz stan pokoju (brudny/czysty). Akcja \code{move} przemieszcza robota, a \code{clean} zmienia stan brudnego pokoju na czysty, jeśli robot w nim jest.

\subsubsection{Problem (\code{problem.pddl})}
\begin{lstlisting}[caption={Problem PDDL dla zadania 2 - Robot Sprzątający}, label={lst:zad2_problem}]
(define (problem robot-move-problem)
    (:domain robot-world)
    (:objects
        pokoj1 pokoj2 pokoj3 - room
        robby - robot
    )
    (:init
        (at robby pokoj1)
        (dirty pokoj1)
        (dirty pokoj2)
        (dirty pokoj3)
    )
    (:goal (and (clean pokoj1) (clean pokoj2) (clean pokoj3)))
)
\end{lstlisting}
\textbf{Opis Problemu:} Trzy pokoje, jeden robot. Początkowo robot jest w \code{pokoj1}, a wszystkie pokoje są brudne. Celem jest, aby wszystkie trzy pokoje były czyste.

\subsection{Zadanie 3: Robot Przenoszący Piłki}
\subsubsection{Domena (\code{domain.pddl})}
\begin{lstlisting}[caption={Domena PDDL dla zadania 3 - Robot Przenoszący Piłki}, label={lst:zad3_domain}]
(define (domain ball-moving-robot)
  (:requirements :strips :typing)
  (:types robot room ball arm)
  (:predicates
    (at ?r - robot ?room - room)
    (inroom ?b - ball ?room - room)
    (holding ?a - arm ?b - ball)
    (arm-empty ?a - arm)
  )
  (:action move
    :parameters (?r - robot ?from - room ?to - room)
    :precondition (at ?r ?from)
    :effect (and (not (at ?r ?from)) (at ?r ?to))
  )
  (:action pick-up
    :parameters (?r - robot ?a - arm ?b - ball ?room - room)
    :precondition (and (at ?r ?room) (inroom ?b ?room) (arm-empty ?a))
    :effect (and (holding ?a ?b) (not (arm-empty ?a)) 
                 (not (inroom ?b ?room)))
  )
  (:action put-down
    :parameters (?r - robot ?a - arm ?b - ball ?room - room)
    :precondition (and (at ?r ?room) (holding ?a ?b))
    :effect (and (inroom ?b ?room) (arm-empty ?a) 
                 (not (holding ?a ?b)))
  )
)
\end{lstlisting}
\textbf{Opis Domeny:} Definiuje typy dla robota, pokoju, piłki i ramienia. Predykaty śledzą pozycję robota i piłek, oraz stan ramion (czy trzymają piłkę, czy są puste). Akcje to \code{move} (ruch robota), \code{pick-up} (podniesienie piłki pustym ramieniem) i \code{put-down} (odłożenie trzymanej piłki).

\subsubsection{Problem (\code{problem.pddl})}
\begin{lstlisting}[caption={Problem PDDL dla zadania 3 - Robot Przenoszący Piłki}, label={lst:zad3_problem}]
(define (problem move-balls)
  (:domain ball-moving-robot)
  (:objects
    room1 room2 - room
    robot - robot
    ball1 ball2 ball3 ball4 - ball
    arm1 arm2 - arm
  )
  (:init
    (at robot room1)
    (inroom ball1 room1) (inroom ball2 room1)
    (inroom ball3 room1) (inroom ball4 room1)
    (arm-empty arm1) (arm-empty arm2)
  )
  (:goal
    (and
      (inroom ball1 room2) (inroom ball2 room2)
      (inroom ball3 room2) (inroom ball4 room2)
    )
  )
)
\end{lstlisting}
\textbf{Opis Problemu:} Dwa pokoje, jeden robot, cztery piłki i dwa ramiona. Początkowo robot i wszystkie piłki są w \code{room1}, a ramiona są puste. Celem jest przeniesienie wszystkich piłek do \code{room2}.

\clearpage
\section{Konfiguracja Eksperymentów i Uruchomienie Planera}
Do rozwiązania zdefiniowanych problemów PDDL wykorzystano system DELFI, który integruje planer Fast Downward. System ten automatycznie wybiera konfigurację planera na podstawie analizy problemu (w tym przypadku z wykorzystaniem modelu uczenia głębokiego, o czym świadczą logi dotyczące TensorFlow).

Dla każdego z trzech zadań, odpowiednie pliki \code{domain.pddl} i \code{problem.pddl} zostały przetworzone przez system. Kluczowe etapy działania planera, widoczne w dostarczonych logach (\code{result.txt}), obejmują:
\begin{enumerate}
    \item \textbf{Parsowanie PDDL:} Wczytanie i zinterpretowanie plików dziedziny i problemu.
    \item \textbf{Normalizacja zadania:} Przekształcenie problemu do wewnętrznej, ustandaryzowanej reprezentacji.
    \item \textbf{Instancjacja:} Wygenerowanie wszystkich możliwych konkretnych predykatów i akcji na podstawie obiektów.
    \item \textbf{Wybór konfiguracji planera:} System DELFI wybrał specyficzne konfiguracje Fast Downward, np. z heurystyką \code{merge_and_shrink} lub \code{cpdbs} i algorytmem przeszukiwania A*. Przykładowe wywołania:
    \begin{itemize}
        \item Zadanie 1: \code{astar(merge_and_shrink(...symmetries=sym,pruning=stubborn_sets_simple...))}
        \item Zadanie 2: \code{astar(merge_and_shrink(...symmetries=sym,pruning=stubborn_sets_simple...))} (inna konfiguracja M\&S)
        \item Zadanie 3: \code{astar(cpdbs(patterns=hillclimbing...),symmetries=sym,pruning=stubborn_sets_simple...))}
    \end{itemize}
    \item \textbf{Przeszukiwanie przestrzeni stanów:} Wybrany algorytm przeszukuje przestrzeń stanów w celu znalezienia sekwencji akcji prowadzącej od stanu początkowego do celu.
    \item \textbf{Generowanie planu:} Jeśli rozwiązanie zostanie znalezione, planer zwraca sekwencję akcji.
\end{enumerate}
Analiza wyników skupi się na wygenerowanym planie oraz metrykach wydajnościowych planera.

\section{Analiza Wyników Eksperymentów}

\subsection{Zadanie 1: Logistyka Transportu Kart Graficznych}
\begin{itemize}
    \item \textbf{Status rozwiązania:} Plan został znaleziony (\code{Solution found!}).
    \item \textbf{Wygenerowany plan:}
    \begin{enumerate}
        \item \code{(load-package gpu1 dron_pwr pwr)}
        \item \code{(load-package gpu2 dron_pwr pwr)}
        \item \code{(move-vehicle dron_pwr pwr agh air)}
        \item \code{(unload-package gpu2 dron_pwr agh)}
        \item \code{(move-vehicle dron_pwr agh pw air)}
        \item \code{(unload-package gpu1 dron_pwr pw)}
    \end{enumerate}
    \item \textbf{Długość/Koszt planu:} 6 kroków / Koszt 6.
    \item \textbf{Komentarz do planu:} Planer optymalnie wykorzystał drona, który zabrał obie karty graficzne z PWr, poleciał do AGH, zostawił tam GPU2, a następnie poleciał do PW, aby zostawić GPU1. Student nie został wykorzystany, co jest logiczne, gdyż dron mógł obsłużyć oba transporty i miał dostęp do wszystkich lokacji drogą powietrzną.
    \item \textbf{Metryki planera:}
    \begin{itemize}
        \item Całkowity czas: 0.127s (wall-clock).
        \item Szczytowe użycie pamięci: 5528 KB.
        \item Stany rozwinięte/ocenione/wygenerowane: 23 / 47 / 127.
        \item Atomy PDDL: 99 relewantnych, Operatory: 36.
    \end{itemize}
\end{itemize}

\subsection{Zadanie 2: Robot Sprzątający}
\begin{itemize}
    \item \textbf{Status rozwiązania:} Plan został znaleziony (\code{Solution found!}).
    \item \textbf{Wygenerowany plan:}
    \begin{enumerate}
        \item \code{(clean robby pokoj1)}
        \item \code{(move robby pokoj1 pokoj2)}
        \item \code{(clean robby pokoj2)}
        \item \code{(move robby pokoj2 pokoj3)}
        \item \code{(clean robby pokoj3)}
    \end{enumerate}
    \item \textbf{Długość/Koszt planu:} 5 kroków / Koszt 5.
    \item \textbf{Komentarz do planu:} Robot kolejno sprząta pokój, w którym się znajduje, a następnie przemieszcza się do następnego brudnego pokoju, aż wszystkie zostaną posprzątane. Jest to najbardziej intuicyjne rozwiązanie.
    \item \textbf{Metryki planera:}
    \begin{itemize}
        \item Całkowity czas: 0.104s (wall-clock).
        \item Szczytowe użycie pamięci: 5528 KB.
        \item Stany rozwinięte/ocenione/wygenerowane: 6 / 6 / 13.
        \item Atomy PDDL: 34 relewantne, Operatory: 9.
    \end{itemize}
    Zauważalnie mniejsza liczba stanów i operatorów w porównaniu do zadania 1, co przekłada się na bardzo szybkie znalezienie prostego planu.
\end{itemize}

\subsection{Zadanie 3: Robot Przenoszący Piłki}
\begin{itemize}
    \item \textbf{Status rozwiązania:} Plan został znaleziony (\code{Solution found!}).
    \item \textbf{Wygenerowany plan:}
    \begin{enumerate}
        \item \code{(pick-up robot arm1 ball1 room1)}
        \item \code{(pick-up robot arm2 ball2 room1)}
        \item \code{(move robot room1 room2)}
        \item \code{(put-down robot arm1 ball1 room2)}
        \item \code{(put-down robot arm2 ball2 room2)}
        \item \code{(move robot room2 room1)}
        \item \code{(pick-up robot arm1 ball3 room1)}
        \item \code{(pick-up robot arm2 ball4 room1)}
        \item \code{(move robot room1 room2)}
        \item \code{(put-down robot arm1 ball3 room2)}
        \item \code{(put-down robot arm2 ball4 room2)}
    \end{enumerate}
    \item \textbf{Długość/Koszt planu:} 11 kroków / Koszt 11.
    \item \textbf{Komentarz do planu:} Robot efektywnie wykorzystuje oba ramiona, przenosząc po dwie piłki na kurs. Wykonuje dwa cykle: podnieś dwie piłki, przenieś do drugiego pokoju, odłóż, wróć po kolejne.
    \item \textbf{Metryki planera:}
    \begin{itemize}
        \item Całkowity czas: 0.139s (wall-clock).
        \item Szczytowe użycie pamięci: 5524 KB.
        \item Stany rozwinięte/ocenione/wygenerowane: 21 / 23 / 81.
        \item Atomy PDDL: 84 relewantne, Operatory: 34.
    \end{itemize}
    Złożoność tego problemu jest pośrednia między zadaniem 1 a 2, co odzwierciedlają metryki.
\end{itemize}

\section{Napotkane Problemy Implementacyjne i Uwagi}
Podczas tworzenia modeli PDDL dla powyższych zadań napotkano kilka typowych wyzwań oraz poczyniono pewne obserwacje:
\begin{itemize}
    \item \textbf{Definiowanie typów i hierarchii:} Kluczowe było precyzyjne zdefiniowanie typów obiektów, szczególnie w Zadaniu 1, gdzie występowały podtypy (np. \code{university - location}, \code{drone - vehicle}). Poprawne typowanie jest niezbędne dla działania planera i unikania błędów.
    \item \textbf{Logika predykatów i akcji:} Staranne formułowanie predykatów oraz warunków wstępnych i efektów akcji było fundamentalne. Błędy na tym etapie mogą prowadzić do niepoprawnych planów lub niemożności znalezienia rozwiązania. Na przykład, w Zadaniu 1, predykat \code{(can-use ?veh ?mode)} oraz warunek \code{(not (= ?from ?to))} w akcji \code{move-vehicle} były istotne dla realizmu modelu.
    \item \textbf{Dwukierunkowość połączeń (Zadanie 1):} W Zadaniu 1, połączenia między lokacjami (np. \code{(connected pwr agh road)}) musiały być zdefiniowane w obie strony (np. również \code{(connected agh pwr road)}), aby umożliwić swobodny ruch pojazdów.
    \item \textbf{Nieokreślona pojemność pojazdów (Zadanie 1):} Model w Zadaniu 1 nie definiował jawnie pojemności pojazdów. Dron mógł "zabrać" obie paczki jednocześnie, co planer wykorzystał. W bardziej realistycznym scenariuszu należałoby dodać predykaty i logikę zarządzania pojemnością.
    \item \textbf{Debugowanie PDDL:} Język PDDL, mimo swojej mocy, może generować trudne do zdiagnozowania błędy, jeśli model jest niepoprawny. Komunikaty z planera (widoczne w \code{result.txt} podczas fazy parsowania i normalizacji) są pomocne, ale czasami wymagają wnikliwej analizy. Użycie edytorów PDDL z walidacją (np. \href{https://editor.planning.domains}{editor.planning.domains}) jest zalecane.
    \item \textbf{Interpretacja wyników planera:} Logi z Fast Downward są szczegółowe i dostarczają wielu informacji o procesie planowania (np. użyte heurystyki, liczba stanów). Zrozumienie tych metryk pozwala ocenić złożoność problemu i efektywność planera.
    \item \textbf{Kreatywność w modelowaniu (Zadanie 1):} Tematyka transportu kart graficznych między politechnikami była próbą stworzenia ciekawszego kontekstu dla standardowego problemu logistycznego, co pokazuje elastyczność PDDL w modelowaniu różnorodnych scenariuszy.
\end{itemize}
Wszystkie problemy zostały ostatecznie rozwiązane, a planery wygenerowały poprawne i logiczne plany.

\section{Podsumowanie i Wnioski}
Realizacja zadań z wykorzystaniem języka PDDL pozwoliła na praktyczne zastosowanie wiedzy z zakresu planowania automatycznego. Udało się pomyślnie zamodelować trzy różne problemy oraz uzyskać dla nich rozwiązania za pomocą planera Fast Downward.
\begin{itemize}
    \item Język PDDL okazał się skutecznym narzędziem do deklaratywnego opisu problemów planowania, umożliwiając abstrakcję od szczegółów implementacyjnych algorytmów przeszukiwania.
    \item Planer Fast Downward, wraz z systemem DELFI, efektywnie znajdował rozwiązania dla przedstawionych problemów, generując logiczne i (w kontekście modeli) optymalne plany działania.
    \item Analiza metryk wydajnościowych (czas, pamięć, liczba stanów) dostarczyła wglądu w relatywną złożoność poszczególnych zadań. Zadanie 1 (logistyka) było najbardziej złożone pod względem liczby atomów i operatorów, podczas gdy Zadanie 2 (robot sprzątający) było najprostsze.
    \item Proces modelowania w PDDL wymaga precyzji i uwagi na detale, szczególnie przy definiowaniu typów, predykatów oraz warunków i efektów akcji.
    \item Wykorzystanie podtypów oraz stałych w PDDL (jak w Zadaniu 1) pozwala na tworzenie bardziej czytelnych i modularnych modeli.
\end{itemize}
Projekt z powodzeniem zrealizował postawione cele edukacyjne, demonstrując praktyczne aspekty tworzenia i rozwiązywania problemów planowania w sztucznej inteligencji. Dalsze kroki mogłyby obejmować eksplorację bardziej zaawansowanych funkcji PDDL, takich jak akcje duratywne, koszty czy planowanie numeryczne, a także testowanie na bardziej złożonych instancjach problemów.

\section*{Materiały Źródłowe}
\begin{itemize}
    \item Materiały wykładowe oraz treść listy zadań nr 3 z kursu "Sztuczna Inteligencja i Inżynieria Wiedzy".
    \item Dokumentacja i przykłady użycia języka PDDL, m.in. z \href{https://planning.wiki/ref/pddl/domain}{planning.wiki}.
    \item Strona edytora online PDDL: \href{https://editor.planning.domains}{https://editor.planning.domains}.
    \item Strona projektu Fast Downward: \href{https://www.fast-downward.org/}{https://www.fast-downward.org/}.
    \item Fares, F. Learn PDDL: \href{https://fareskalaboud.github.io/LearnPDDL/}{https://fareskalaboud.github.io/LearnPDDL/}.
    \item University of Toronto, CS2542. Introduction to PDDL: \href{https://www.cs.toronto.edu/~sheila/2542/s14/A1/introtopddl2.pdf}{https://www.cs.toronto.edu/~sheila/2542/s14/A1/introtopddl2.pdf}.
    \item PDDL4J Tutorial: \href{http://pddl4j.imag.fr/pddl_tutorial.html}{http://pddl4j.imag.fr/pddl\_tutorial.html}.
\end{itemize}

\end{document}