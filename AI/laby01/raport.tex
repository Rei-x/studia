\documentclass[12pt,a4paper]{article}

% Essential packages
\usepackage[utf8]{inputenc}
\usepackage[T1]{fontenc}
\usepackage[polish]{babel}
\usepackage{graphicx}
\usepackage{amsmath,amssymb}
\usepackage{algorithm}
\usepackage{algorithmic}
\usepackage{booktabs}
\usepackage{hyperref}
\usepackage{url}
\usepackage{listings}
\usepackage{xcolor}
\usepackage{geometry}

% Set page margins
\geometry{a4paper, margin=2.5cm}

% Define colors for code listings
\definecolor{codegreen}{rgb}{0,0.6,0}
\definecolor{codegray}{rgb}{0.5,0.5,0.5}
\definecolor{codepurple}{rgb}{0.58,0,0.82}
\definecolor{backcolour}{rgb}{0.95,0.95,0.92}

% Configure code listings style
\lstdefinestyle{mystyle}{
    backgroundcolor=\color{backcolour},   
    commentstyle=\color{codegreen},
    keywordstyle=\color{magenta},
    numberstyle=\tiny\color{codegray},
    stringstyle=\color{codepurple},
    basicstyle=\ttfamily\footnotesize,
    breakatwhitespace=false,         
    breaklines=true,                 
    captionpos=b,                    
    keepspaces=true,                 
    numbers=left,                    
    numbersep=5pt,                  
    showspaces=false,                
    showstringspaces=false,
    showtabs=false,                  
    tabsize=2
}
\lstset{style=mystyle}

% Document info
\title{\LARGE \textbf{Problem Komiwojażera w Systemie Komunikacji Miejskiej}\\
\large Algorytmy optymalizacyjne w zastosowaniu do planowania tras}
\author{Imię Nazwisko}
\date{\today}

\begin{document}

\maketitle

\begin{abstract}
    Niniejszy raport przedstawia implementację i analizę algorytmów służących do optymalizacji tras w systemie komunikacji miejskiej. W pierwszej części opisano algorytmy wyszukiwania najkrótszych połączeń między przystankami (algorytm Dijkstry oraz A*) przy uwzględnieniu różnych kryteriów optymalizacji: czasu przejazdu oraz liczby przesiadek. W drugiej części przedstawiono rozwiązanie problemu komiwojażera dla odwiedzenia zbioru przystanków z wykorzystaniem metody przeszukiwania z zabronieniami (Tabu Search) wraz z modyfikacjami mającymi na celu poprawę jakości rozwiązań.
\end{abstract}

\tableofcontents
\newpage

\section{Wprowadzenie}
\subsection{Opis problemu}
W niniejszym raporcie zajmujemy się dwoma zadaniami optymalizacyjnymi w kontekście planowania tras w komunikacji miejskiej:
\begin{enumerate}
    \item Wyszukiwanie najkrótszych połączeń między dwoma przystankami z uwzględnieniem kryteriów czasu przejazdu lub liczby przesiadek.
    \item Planowanie optymalnej trasy przejazdu przez zbiór zadanych przystanków z powrotem do punktu początkowego (wariant problemu komiwojażera).
\end{enumerate}

\subsection{Dane wejściowe}
Dane dotyczące systemu komunikacji miejskiej dostarczone są w pliku \texttt{connection\_graph.csv}. Plik ten zawiera informacje o połączeniach między przystankami, czasach przejazdów oraz liniach komunikacyjnych.

\section{Zadanie 1: Wyszukiwanie najkrótszych połączeń}
\subsection{Algorytm Dijkstry}
\subsubsection{Opis teoretyczny}
Algorytm Dijkstry służy do znajdowania najkrótszych ścieżek w grafie ważonym o nieujemnych wagach krawędzi. W kontekście naszego zadania, graf reprezentuje sieć transportu publicznego, gdzie wierzchołkami są przystanki, a krawędzie oznaczają bezpośrednie połączenia między nimi.

\subsubsection{Implementacja dla optymalizacji czasu przejazdu}
% Opis implementacji algorytmu Dijkstry do minimalizacji czasu przejazdu

\subsubsection{Wyniki i analiza}
% Wyniki eksperymentów, przykładowe trasy, porównanie z innymi algorytmami

\subsection{Algorytm A* dla optymalizacji czasu przejazdu}
\subsubsection{Opis teoretyczny}
Algorytm A* stanowi rozszerzenie algorytmu Dijkstry poprzez wprowadzenie heurystyki, która szacuje koszt dotarcia do celu. W przypadku optymalizacji czasu przejazdu, heurystyka może opierać się na odległości euklidesowej podzielonej przez średnią prędkość środków transportu.

\subsubsection{Implementacja}
% Opis implementacji A* dla minimalizacji czasu przejazdu

\subsubsection{Wyniki i analiza}
% Wyniki eksperymentów, porównanie z algorytmem Dijkstry

\subsection{Algorytm A* dla optymalizacji liczby przesiadek}
\subsubsection{Opis teoretyczny}
W tym wariancie funkcja kosztu oraz heurystyka są dostosowane do minimalizacji liczby przesiadek między liniami komunikacyjnymi.

\subsubsection{Implementacja}
% Opis implementacji A* dla minimalizacji liczby przesiadek

\subsubsection{Wyniki i analiza}
% Wyniki eksperymentów, porównanie z poprzednimi implementacjami

\subsection{Modyfikacje algorytmu A*}
\subsubsection{Opis wprowadzonych modyfikacji}
% Opis modyfikacji zwiększających efektywność algorytmu

\subsubsection{Wyniki i porównanie z podstawową wersją}
% Porównanie efektywności zmodyfikowanego algorytmu z wersją podstawową

\section{Zadanie 2: Problem odwiedzenia zbioru przystanków}
\subsection{Metoda przeszukiwania z zabronieniami (Tabu Search)}
\subsubsection{Opis teoretyczny}
Tabu Search to metaheurystyka wykorzystywana do rozwiązywania problemów optymalizacyjnych. W kontekście problemu komiwojażera, metoda ta pozwala na efektywne przeszukiwanie przestrzeni rozwiązań z uniknięciem utknięcia w lokalnym optimum.

\subsubsection{Podstawowa implementacja}
% Opis implementacji podstawowej wersji Tabu Search

\subsubsection{Wyniki i analiza}
% Wyniki eksperymentów dla podstawowej wersji algorytmu

\subsection{Modyfikacja długości listy tabu}
\subsubsection{Opis teoretyczny}
Długość listy tabu ma istotny wpływ na efektywność algorytmu. Zbyt krótka lista może prowadzić do cyklicznego przeszukiwania tych samych rozwiązań, natomiast zbyt długa może nadmiernie ograniczać przestrzeń poszukiwań.

\subsubsection{Implementacja}
% Opis implementacji adaptacyjnego doboru długości listy tabu

\subsubsection{Wyniki i analiza}
% Porównanie z wersją podstawową

\subsection{Modyfikacja z kryterium aspiracji}
\subsubsection{Opis teoretyczny}
Kryterium aspiracji pozwala na przyjęcie rozwiązania znajdującego się na liście tabu, jeśli spełnia ono określone warunki, np. jest lepsze od najlepszego dotychczas znalezionego rozwiązania.

\subsubsection{Implementacja}
% Opis implementacji kryterium aspiracji

\subsubsection{Wyniki i analiza}
% Porównanie z poprzednimi wersjami algorytmu

\subsection{Strategia próbkowania sąsiedztwa}
\subsubsection{Opis teoretyczny}
Efektywne próbkowanie sąsiedztwa bieżącego rozwiązania może znacząco wpłynąć na wydajność algorytmu, szczególnie dla dużych instancji problemu.

\subsubsection{Implementacja}
% Opis implementacji strategii próbkowania sąsiedztwa

\subsubsection{Wyniki i analiza}
% Porównanie z poprzednimi wersjami algorytmu

\section{Podsumowanie}
\subsection{Porównanie zaimplementowanych algorytmów}
% Porównanie wszystkich zaimplementowanych algorytmów pod kątem jakości rozwiązań i czasu obliczeń

\subsection{Napotkane problemy implementacyjne}
% Opis problemów napotkanych podczas implementacji oraz sposobów ich rozwiązania

\subsection{Wnioski}
% Ogólne wnioski dotyczące efektywności algorytmów w kontekście planowania tras w komunikacji miejskiej

\section{Wykorzystane biblioteki}
% Opis bibliotek wykorzystanych do implementacji algorytmów

\begin{thebibliography}{9}
    \bibitem{dijkstra} Dijkstra, E. W. (1959). A note on two problems in connexion with graphs. Numerische Mathematik, 1(1), 269-271.
    \bibitem{astar} Hart, P. E., Nilsson, N. J., & Raphael, B. (1968). A formal basis for the heuristic determination of minimum cost paths. IEEE Transactions on Systems Science and Cybernetics, 4(2), 100-107.
    \bibitem{tabu} Glover, F. (1989). Tabu search—part I. ORSA Journal on Computing, 1(3), 190-206.
    \bibitem{tsp} Lawler, E. L., Lenstra, J. K., Rinnooy Kan, A. H. G., & Shmoys, D. B. (1985). The Traveling Salesman Problem: A Guided Tour of Combinatorial Optimization. Wiley.
\end{thebibliography}

\appendix
\section{Kod źródłowy}
% Fragmenty kodu źródłowego ilustrujące kluczowe elementy implementacji

\section{Przykładowe dane wejściowe i wyjściowe}
% Przykłady danych wejściowych i odpowiadających im wyników

\end{document}
