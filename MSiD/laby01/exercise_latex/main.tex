\documentclass{article}
\usepackage{amsmath} 
\usepackage{hyperref}
\usepackage{graphicx}
\usepackage{multirow}
\usepackage[table,xcdraw]{xcolor}
\usepackage[
backend=biber,
style=alphabetic,
sorting=ynt
]{biblatex}
\addbibresource{sample.bib}
\title{Dokumencik}
\author{Bartosz Gotowski}

\begin{document}
\maketitle

\tableofcontents

\section{Sekcja (ale nie sportowa)}

\subsection{podsekcja}

\subsubsection{podpodsekcja}

This introductory tutorial does not assume any prior experience of LaTeX but, hopefully, by the time you are finished, you will not only have written your first LaTeX document but also acquired sufficient knowledge and confidence to take the next steps toward LaTeX proficiency.
What is LaTeX?

LaTeX (pronounced “LAY-tek” or “LAH-tek”) is a tool for typesetting professional-looking documents. However, LaTeX’s mode of operation is quite different to many other document-production applications you may have used, such as Microsoft Word or LibreOffice Writer: those “WYSIWYG” tools provide users with an interactive page into which they type and edit their text and apply various forms of styling. LaTeX works very differently: instead, your document is a plain text file interspersed with LaTeX commands used to express the desired (typeset) results. To produce a visible, typeset document, your LaTeX file is processed by a piece of software called a TeX engine which uses the commands embedded in your text file to guide and control the typesetting process, converting the LaTeX commands and document text into a professionally typeset PDF file. This means you only need to focus on the content of your document and the computer, via LaTeX commands and the TeX engine, will take care of the visual appearance (formatting).

Arguments in favour of LaTeX include:

    support for typesetting extremely complex mathematics, tables and technical content for the physical sciences;
    facilities for footnotes, cross-referencing and management of bibliographies;
    ease of producing complicated, or tedious, document elements such as indexes, glossaries, table of contents, lists of figures;
    being highly customizable for bespoke document production due to its intrinsic programmability and extensibility through thousands of free add-on packages.

Overall, LaTeX provides users with a great deal of control over the production of documents which are typeset to extremely high standards. Of course, there are types of documents or publications where LaTeX doesn’t shine, including many “free form” page designs typically found in magazine-type publications.

\section{Style tekstów}

\textit{Ten tekst jest w kursywie}, \underline{ten podkreślony}, \textbf{ten wytłuszczony}, aaaaa tamten \textbf{\textit{\underline{MA WSZYSTKO NA RAZ!}}}

\section{Środowiska}

\subsection{Listy}

Lista też
\begin{itemize}
    \item to
    \item jest
    \item lista
\end{itemize}

Listaaa

\begin{enumerate}
    \item to
    \item jest
    \item lista
\end{enumerate}

\subsection{Równania}
ruwnanie $\Delta = b^2 - 4ac $, pełne równanie
\[
\sum_{n=1}^{5} n = 1 + 2 + 3 + 4 + 5 = 15 \label{eq:sum}
\]

odnośnik do równania \ref{eq:sum}.

\subsection{Tabelki}

Tabelki1!!

\begin{table}[]
\centering
\begin{tabular}{|l|llll|ll|l|
>{\columncolor[HTML]{FFCC67}}l |
>{\columncolor[HTML]{FFCC67}}l |}
\hline
 &
  \multicolumn{1}{l|}{} &
  \multicolumn{1}{l|}{} &
  \multicolumn{1}{l|}{} &
   Whaaat &
  \multicolumn{2}{l|}{} 
   & 
   & 
   \\ \cline{1-7}  \cline{9-10} 
 &  
  \multicolumn{4}{l|}{} &
  \multicolumn{1}{l|}{{\color[HTML]{68CBD0} }} &
  {\color[HTML]{68CBD0} } &
  \multirow{-2}{*}{} & 
  {\color[HTML]{68CBD0} } &
   \\ \hline 
\end{tabular}
\label{table:Crazy tabelka}
\caption{Crazy tabelka}
\end{table}

Przykładowy generator tabel: \url{https://www.tablesgenerator.com} \\(do 
stworzenia linka użyłem paczki 'hyperref' oraz tagu url).

\subsection{Obrazki}
\begin{figure}[h]
    \centering
    \includegraphics[width=0.5\linewidth]{frog.jpg}
    \caption{Żabkaaa from \cite{hh2010a}}
    \label{fig:enter-label}
\end{figure}

żabkaaa

\printbibliography
\end{document}